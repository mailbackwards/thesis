\chapter{Tracing the Links}

\section{History/theory of link analysis}

\section{Link breakdown by category}

\section{Link breakdown by graph/network}

% This section will have quantitative analysis of inlinking vs. outlinking, both historical and my own research.
% Maybe call out 1-2 particular cases (NYT? Guardian? etc.) to see how they're linking

The last chapter dealt with how publishers think about their archives. This section focuses instead on how they're \emph{linking}, through research and quantitative link analysis. It will combine historic news network analyses and a study that I performed with the help of Media Cloud.


%Even once we assume that linking is a fundamental good for journalists, the problem remains of standardizing and developing a language \emph{for} linking, for this new affordance in citation and context provision.

% Anthony Grafton %  % Google books / Google scholar % % This should all probably be in a different place though %

%The digital realm, unbound by physical juxtapositions, can not only add an arbitrary number of links, but it can link \emph{in both directions}. Where paper books can only take a reader back in time, pointing only to its sources, it cannot tell you who used it as a source in turn. This results in what network scientists call a ``directed'' network, rather than a bidirectional one. to adds to these links.

% Jaron Lanier: "You're throwing away all the best parts of a network" (or Weinberger? Barabasi?). "Could not have greater implications"

% Adamic and others have tracked links across blogospheres and social networks. They rarely turn inward to a particular organization and see what they have to say.
% Deep examination of inlinks vs. outlinks. Usually Hyperlink Network Analysis specifically throws out the inlinks. So do search engines; called "nepotistic" links. This on the other hand hones in on them.
% Emphasize that I'm just using the content within the text of the article itself; there is much more. Anne Helmond "Exploring the Boundaries of a Website"; "The News Reads Us"
% Hyperlinking as Gatekeeping -- 2003 study of Timothy McVeigh. "They are unlikely to offer external hyperlinks." So early on it was especially true. Go through that study as an example. 95% of all links were inlinks. Over 96% of these links were from a sidebar! 0.6% from the text itself. Though keep in mind: the web was very very different then. Tabs weren't really a thing so people were more worried about losing the reader or the reader losing their place. Websites were also less dynamic, which allowed for more consistent coding of related article sidebars. As this gets more customized/personalized, it's hard to use these as strong signals for analysis. In 2003 most papers were at the 2nd stage of digital news development.
% Related to sidebar discussion: there's a difference between blogroll and in-post links in blogs. De Maeyer 2013 p. 745
% More than this: Helmond MIT8 2013 -- New York Times pages have dozens of JavaScript trackers going, linked to other websites and services.
% Cyberbalkanization (Sunstein 2001) -- massification -- gatekeeping/walled garden of hyperlinks. "gated cybercommunities" (Tremayne 2005); "walled gardens" (Napoli 2008). Does this also apply in-domain? Can we see rifts among news desks, for instance? Places where they fail to categorize, read, or cite one another?
% Vobic Slovenian newspaper study: even in 2013 many of these orgs are not thinking forward about hypertext
% Does the hyperlink go against certain ad models? If your goal is to encourage clicks to advertisers, adding more hyperlinks to a page theoretically reduces the probability that the user will navigate to an ad.
% BUT: while networks have mostly been studied in terms of outlinks, there have also been many WIF studies that are internal to a given domain, e.g. the Adamic study that looked at communities at Stanford and MIT, or this "Macroscopic Information Web Links" paper that examines impact within a university (inlinks)


Given these potentials, it might be surprising to find that many publishers are very reticent to link. Those who do have linking policies are often quite conservative.

% OVERVIEW OF LINKING POLICIES: NYT, BBC, Guardian, etc. % % Reference to next section where I explain more what they're doing %

Some of this is due to SEO; while no one knows exactly what Google's mercurial PageRank algorithm is doing, it's clear that links form a fundamental component (and as critics such as Clay Shirky have argued, relying on links rather than traditional categories and tags has been the crux of their success over competitors).\autocite{} Publishers are also wary of taking a reader away from their own website. But I'm also sure that much of the fear of links comes from inertia and tradition; since journalists never used to have a way to link, some don't see a need to start.

While many have debated the potentials and pitfalls of hyperlinking the news, I am proposing an additional wrinkle to the conversation; a smart use of linking can be, to borrow Derrida's term, a \emph{pledge} to better structure the news and keep archives continuously animated and relevant.


% \section{Mapping Links}

% There is a lot of research on automatically mapping hyperlink networks; crawling, clustering, classifying utilizing a variety of machine learning techniques. However, most of these studies are interested in the ecosystem of content creators, and the ways in which links form across domains. My focus is instead on \emph{internal} links between stories. This is not only a more manageable set, but it is also allows opportunities for rich restructuring of the data schema. Chakrabarti says: "I believe that the Web will always remain adventurous enough that its most interesting and timely content can never be shoehorned into rigid schemata." This may be true, but internally, news publishers can use ankle-deep semantics to greatly improve their structures.

% Some webometrics endeavors have aimed to measure internal links within businesses, universities, etc. Others have done much work with internal clickstreams in order to gain insight about website usability. But nothing has looked at internal links of news stories.

% Some theory about how to do it. Talk to César? Reference De Maeyer's piece about it.
% Mapping, but also visualizing
% End by explaining my methodology

% De Maeyer "critical review of link studies". There are many limitations to webometrics. For instance, over-reliance on the information in the URL domain name and path, which we all agreed was not a uniform, consistent resource. This led to, e.g. Tuvalu being among the main hubs of developed countries in hyperlink network, because they sell their .tv domain.
% Maps of the blogosphere were popular in Web 2.0, they were "a link-driven genre" (De Maeyer 2013); but now we're in the era of platforms, how has that changed the link structure of the web?
% Also, a case against straight statistics and confidence intervals: "Links are not statistically independent observations, ‘because of factors such as imitation between web authors (…), the copying of pages or parts of pages, and automatic creation of web pages by server software’ (Thelwall, 2006: 7). According to Thelwall, this means that inferential statistics, such as confidence intervals, are not valid. As Harries et al. (2004) phrase it, web link creation is a ‘social activity that inspires imitation, the opposite of statistical independence’ (Harries et al., 2004: 439)."

% The goal, then: "semi-automatic methods". Most projects "rely on some manual coding, qualitative appreciations and human expertise." The main question is "when do we stop to purely describe the structure of links as technical objects and when do we start to relevantly exploit links to make sense of a social phenomenon?" But ultimately "mixed methods appropriately allow using hyperlinks as proxies for other phenomena." --  "combining quantitative link counts, qualitative inquiries and valuation of field expertise to support link interpretation."

% Fragoso 2011 tries to holistically combine the quantiative/qualitative, macro/micro study patterns that De Maeyer observes: "Web Science has favoured macroscopic approaches which have revealed much about the Web’s structural patterns. We argue that contextualised knowledge about hyperlinks on the Web has not advanced at the same rate and that complementary intermediate and micro-scale investigations are essential for a better understanding of the motivations, functions and meanings of these links."

% Could combine Media Cloud aggregate ``big data'' view with Times Newswire API (e.g. 1 month, ~9k stories) for deeper view into one publication. Deeper view could be a) anchor text, b) where the link is in relation to the top of the article, named entities, etc.

% \section{Automated NER and linked data potential}

% Can we get richer info automatically out of the entities referenced in newsrooms?

% Can we form networks out of the links that newspapers are making? Clusters, in network terms? How are they improvements on classifications, in a more traditional sense? a) they don't need to take as much account of the media type of the resource, right? Text, image, video, etc. are all treated semi-equally.

% Visualization of hyperlink networks: De Maeyer "Methods for Mapping Hyperlink Networks", Carriere/Kazman. De Maeyer "The web lacks geography," it's flat
% ``hyperlink maps are thus helpful in visualizing phenomena that would have been otherwise hard to detect: "the reason for performing such an analysis is to reveal latent structures that are not already obvious to an observer''


