
The hyperlink on the web consists essentially of three a URL (the ``target'') The URL is treated as a unique identifier and as an atomic unit of information, when it's truly neither.

\begin{description}
\item[The URL is not unique] The same article or event listing reappears under dozens of URLs, and any attempts to find a ``canonical URL'' are expensive and inconsistent. Sometimes -- like in the case of a wire service that gets aggregated by several publishers -- there's no singular home for it.

\item[The URL is not atomic] URLs point to multitudes of resources, or none at all. Links have text, pictures, videos, audio, other links, and annotations on all of the above. They might change depending on who's asking for them and when they're asking. They might give your computer a virus. They can (and often do) cause money to change hands between unseen actors. They can open in a new tab or window, or open your email client.
\end{description}

In other words, the \emph{Uniform Resource Locator} is not uniform, nor is it necessarily a resource. The rest of the link (the <a href="..."\> and everything in between) doesn't give many clues as to what's behind it, or what the significance of the connection is. As a software developer, this leaves me scrambling for any clues as to how to define the content and its relationship to other content; aside from clicking the link and visiting the resource, which is slow and expensive at scale, you can't really know what's behind a link. 


