\section{Preface}

% %% Personal background

% For the past two years I've been doing research online about the %problems, challenges, and
% perils of doing research online. This has made my head spin more than once. It is a slippery subject; I keep running into the very problems I want to address. I've encountered so many tantalizing online resources, only to discover the dreaded 404 NOT FOUND. My note repositories and reference lists have ballooned to unmanageable sizes. I've shared frustrations and frequent discussions with my colleagues about the best ``tools'' for organizing our resources and ideas. And I've spent sleepless nights trying to organize my thoughts, and make connections between everything I read, process, and understand. I want my notes and citations to reflect, enhance, and expand my own memories and ideas; too often they obfuscate and distract from them instead. Computers are very good at storing and remembering information, but they are less adept at making connections between the bits of data that they remember.

% %% Quick theorizing/summary

% This is a problem on a collective as well as personal level; it affects not only personal memory, but collective history. We are overloaded with information on a massive, unprecedented scale, and new material arrives faster than we can contain it. For centuries, librarians and archivists collected and sorted out our history and access to the past. Now the majority of our personal history is collected and sorted by newer and less tried methods for determining what something is about, whether it's worth saving, and whether it has any meaningful impact on the world. And the librarians and archivists aren't the main ones doing the sorting.

% The web has enabled an unprecedented explosion of data, and an unprecedented amount of access to it. Another way to frame this is that the past -- the archive -- is bigger and more accessible than ever before. It's an exciting prospect, but the archive has exploded and networked to an unmanageable scale. Machines help to sort out the results, but the best machines don't treat the web as an archive; they treat it as a network. Instead of using categories, they rely on links.

% The link is an ideally situated entity for the post-deconstruction, networked age. There is no hierarchy in a network, only a collection of nodes and links.\footnote{However, there are centers in a network.} Unlike in a library, bookstore, department store, or anywhere that contains physical \emph{things} there is no traditional category; no singular, fixed decision made about what something \emph{is}, what it means, or where it belongs. Instead machines look for what and where something \emph{points to}, and let the links sort everything out. Links serve a double function: you not only see who is linking, but how many links there are. Links not only categorize, they measure importance and impact.

% %% Experience as software engineer

% I especially notice the powers of links in my work as a software engineer and backend web developer. I've built a variety of news and event curation and monitoring applications, using many different programming languages and frameworks.\footnote{e.g. for MIT HyperStudio (\emph{Artbot}, Ruby on Rails), Nieman Journalism Lab (\emph{Fuego}, Flask), and Wiser (Django).} I am essentially a Link Wrangler. I corral articles, emails, events, tweets, and the like in order to classify and ultimately rank them for users. But I have grown frustrated by the link. Links tend to be the unique identifier for a resource, and an atomic unit of information. But links are more elusive and complicated than that; they contain multitudes, and are aggregations themselves. I'm often frustrated by the link's limitations in defining, classifying, and measuring online content.

% %% Goal

% In this thesis I aim to unpack what links do -- and what they fail to do -- for developers, creators, publishers, aggregators, and everyday users of the web. In doing so, I hope to elucidate the ways that information becomes knowledge, news becomes history, and the archive unfolds in a hyperlinked environment. I want to bridge the ways links affect public discourse and cultural memory. % or more production-oriented than that: news production and archive creation?
% My goal is to speak on one hand to news and media organizations, to enact and enliven their own archives and research tools, and on the other hand to libraries and archives, to inject historical resources and context into current events and social issues.

% In the process, I hope to help define the borders and the limitations of the web in particular, and networks in general; what's exciting and new about big data, and what's risky. The link is a smaller and more manageable entity than the network, so it deserves more exploration to see what the smaller unit can teach us about the bigger picture. Some other writings aim to teach the reader how to ``think networks''. I wonder if it might be easier and healthier to ``think links''.

% % help people envision new ways to make connections
% % some business lessons in it-- don't go in all-encompassing
% % mostly: help navigate what's exciting about big data and what's risky about it
% % In short, I hope to link news and history, and help enliven digital repositories with new connections and

% The link might seem like an abstract, academic or even trivial thing, one that is too academic or tangential to real industries like news and libraries. But ``links give power.'' They are the foundation of the web, and they serve as the battleground for much of its political economy. % sentence about why this is relevant HAHAHA
% The goal is to keep my focus not on any one medium or industry, but instead on the nature, the identity, and the mechanics of comparison, difference, and connection.\footnote{One could say that I've taken the ``comparative'' part of Comparative Media Studies too seriously.}

% Whenever anyone blogs about, emails, tweets, likes, or searches for a resource, that resource is recalibrated, recategorized and re-measured. So in one sense, we're all archivists: we constantly save, edit, and delete our traces on emails, files, and social media--and this in turn affects what others will see. We make links, and links make history. But the web is no traditional archive; it's a cloud, not a vault. And the archive is in new hands, where we can't determine or even know the rules under which information has influence, and we can't opt out.

% % also maybe: explain why I didn't create my own publishing platform or utilize links myself %
