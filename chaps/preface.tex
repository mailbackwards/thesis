\section{Preface}

% %% Personal background

For the past two years, I've been doing research online about the problems with doing research online. This has made my head spin more than once. I keep running into the very problems I want to address. I've discovered so many tantalizing online resources, only to encounter the dreaded \texttt{404 NOT FOUND}. My note repositories and reference lists have ballooned to unmanageable sizes. I've shared frustrations and frequent discussions with my colleagues about the best tools and techniques for organizing resources and ideas. And I've spent countless hours trying to collect my own thoughts, and make connections between everything I read, process, and understand. Computers are very good at helping us store and remember information, but they are less adept at making connections between the bits of data that they remember. I want my notes and citations to reflect and expand on my own memories and ideas; too often they obfuscate and distract from them instead.

% %% Quick theorizing/summary
I spent the summer in between these two years at Harvard's Nieman Journalism Lab, and here I turned my attention towards the digital journalism sphere. Beleaguered graduate students and academics are not the only ones facing the perils of online research; the modern journalist is increasingly playing the role of online researcher and information filterer, too. As amplifiers of important stories and topics for public discussion, journalists form a crucial bridge between the individual and the collective; their information problems affect their readers' knowledge and understanding in turn. This also highlights how online actors---publishers, information technology companies, and cultural heritage institutions alike---are fighting for readers' attention and preserving knowledge for the future.

For centuries, librarians and archivists collected and sorted out our history and access to the past. Now the majority of our personal history is collected and sorted by newer, more automated, and less tested methods for determining what something is about and whether it's worth saving. And the librarians and archivists aren't the main ones doing the sorting. Some of these developments offer great potential and excitement; the web has enabled an explosion of data and access to it. The past is bigger and closer to us than ever before. This is an exciting prospect, and much of today's utopian rhetoric around big data and information access draws from such potential; but the landscape is new and the actors involved often operate in secrecy. This extends beyond a journalist or academic conducting efficient research; these individual-scale problems combine to affect our collective understanding of history.

% Maybe something here about the tool and my experience as a software dev?

This thesis is, perhaps ironically, a mostly linear document about hypertext. This is not to say that each chapter must be read completely and sequentially; some chapters might be more pertinent to publishers, journalists, and newsmakers, while others may resonate with information theorists and digital humanists. But I endeavor to make the whole greater than the sum of its parts, and my goal is to highlight the overlaps between these worlds, which are each enabled and threatened by the same technological and economic forces. So this linear thesis hopes to bring these camps together in a single piece rather than encouraging fragmentary readings down familiar pathways. When Ted Nelson republished his famous \emph{Literary Machines}, he lamented: ``It should have been completely rewritten, of course, but various people have grown to love its old organization, so I have grafted this edition's changes onto the old structure.'' Perhaps this thesis should also be completely rewritten, and structured as hypertext; but for now, I will merely aim to graft its connections and diversions onto the usual format, and cohere into a linear structure that I will leave for future efforts to take apart.
%\autocite[1.1 (CHECK THIS)]{nelson_literary_1987}

% Machines help to sort out the results, but the best machines don't treat the web as an archive; they treat it as a network. Instead of using categories, they rely on links.

% The link is an ideally situated entity for the post-deconstruction, networked age. There is no hierarchy in a network, only a collection of nodes and links.\footnote{However, there are centers in a network.} Unlike in a library, bookstore, department store, or anywhere that contains physical \emph{things} there is no traditional category; no singular, fixed decision made about what something \emph{is}, what it means, or where it belongs. Instead machines look for what and where something \emph{points to}, and let the links sort everything out. Links serve a double function: you not only see who is linking, but how many links there are. Links not only categorize, they measure importance and impact.

% %% Experience as software engineer

% I especially notice the powers of links in my work as a software engineer and backend web developer. I've built a variety of news and event curation and monitoring applications, using many different programming languages and frameworks.\footnote{e.g. for MIT HyperStudio (\emph{Artbot}, Ruby on Rails), Nieman Journalism Lab (\emph{Fuego}, Flask), and Wiser (Django).} I am essentially a Link Wrangler. I corral articles, emails, events, tweets, and the like in order to classify and ultimately rank them for users. But I have grown frustrated by the link. Links tend to be the unique identifier for a resource, and an atomic unit of information. But links are more elusive and complicated than that; they contain multitudes, and are aggregations themselves. I'm often frustrated by the link's limitations in defining, classifying, and measuring online content.

% %% Goal

% In this thesis I aim to unpack what links do -- and what they fail to do -- for developers, creators, publishers, aggregators, and everyday users of the web. In doing so, I hope to elucidate the ways that information becomes knowledge, news becomes history, and the archive unfolds in a hyperlinked environment. I want to bridge the ways links affect public discourse and cultural memory. % or more production-oriented than that: news production and archive creation?
% My goal is to speak on one hand to news and media organizations, to enact and enliven their own archives and research tools, and on the other hand to libraries and archives, to inject historical resources and context into current events and social issues.

% In the process, I hope to help define the borders and the limitations of the web in particular, and networks in general; what's exciting and new about big data, and what's risky. The link is a smaller and more manageable entity than the network, so it deserves more exploration to see what the smaller unit can teach us about the bigger picture. Some other writings aim to teach the reader how to ``think networks''. I wonder if it might be easier and healthier to ``think links''.

% % help people envision new ways to make connections
% % some business lessons in it-- don't go in all-encompassing
% % mostly: help navigate what's exciting about big data and what's risky about it
% % In short, I hope to link news and history, and help enliven digital repositories with new connections and

% The link might seem like an abstract, academic or even trivial thing, one that is too academic or tangential to real industries like news and libraries. But ``links give power.'' They are the foundation of the web, and they serve as the battleground for much of its political economy. % sentence about why this is relevant HAHAHA
% The goal is to keep my focus not on any one medium or industry, but instead on the nature, the identity, and the mechanics of comparison, difference, and connection.\footnote{One could say that I've taken the ``comparative'' part of Comparative Media Studies too seriously.}

% Whenever anyone blogs about, emails, tweets, likes, or searches for a resource, that resource is recalibrated, recategorized and re-measured. So in one sense, we're all archivists: we constantly save, edit, and delete our traces on emails, files, and social media--and this in turn affects what others will see. We make links, and links make history. But the web is no traditional archive; it's a cloud, not a vault. And the archive is in new hands, where we can't determine or even know the rules under which information has influence, and we can't opt out.

% % also maybe: explain why I didn't create my own publishing platform or utilize links myself %
