\chapter{Publishers and the Archive}

% This section will have interviews and stuff

In the previous sectionse I have outlined the ways in which the archive, and critical notions of it, has shifted from a fixed and graspable entity to a suite of interconnected parts, constantly shifting and morphing and adapting to new information. This chapter hones in specifically on the newsroom archive, and the ways in which online publishers and legacy news outfits are dealing with their digital and digitized archives.

News publishers prove an ideal study for examining the potentials of hypertext archives. If we treat a newspaper as a proto-hypertextual document, it becomes apparent that online news might be a natural extension of reading the newspaper. Few readers go through a newspaper sequentially, paying equal attention to every article; instead the reader jumps around from page to page, skimming some sections for its raw information while reading longer pieces more deeply. A newspaper's front page reads like a website homepage, with snippets and teasers that aim to draw the reader deeper. A given page can hold several articles, and an interested reader might be distracted or intrigued by a ``related article'' next to the one he or she came to read. Some works are categorized into sections -- arts, sports, letters to the editor -- while others might be paired with a certain advertisement or reaction article. These examples point to the inherently interlinked nature of newspapers; they might seem to naturally lend themselves to digitization.

The online space, unbound by physical juxtapositions, only adds to these links. A paper might reference a major news event, a book, or another article far from the page, and it is here that digital archives create potential for news.

\section{The networked news archive}

The first stage for any legacy publisher -- anyone that creates physical newspapers or magazines -- is to \emph{digitize} the archive. This tends to consist of scanning the pages of old publications, running OCR (optical character recognition) on each page, and exposing the results to a search interface for researchers and interested readers.

Most publishers have reached this stage; it is a crucial first step for enlivening the archive, but a physical record can often limit the digital equivalent's potentials. Digital versions of physical articles often do not leverage links and mixed media to the same effect. While a digital-native version of a print article might directly cite more sources or feature an intriguing interactive, these elements remain second-class citizens to the print article, which digital versions must remain faithful to.

It is also telling that many of the digitization projects, begun decades ago, focused exclusively on salvaging the text. This ignores substantial information in the archive; especially images, which have long been seen as a way to monetize archives. % Add aside here about how images could be monetized? %
Some publishers have re-scanned their entire archive in order to capture the images that they ignored years ago.  % The NYT advertisement search % % Lily tried this; also that guy who went through and just got the images from old sources %

The second stage is to \emph{atomize} the archive; to break these scanned pages into their consituent parts. Given the newspaper's inherently hypertextual nature, this is a major challenge at any scale. What metadata is worth saving? The text, the subtext, the pictures? The photo or pullquote on the side?

I have been using the term ``linked'' or ``networked'' to describe the archive; but when legacy news publishers refer to a ``linked'' record in a digital archive, they are referring to the ability to link back to the scanned original source, where a database record can lead to a view of the ``original'' in PDF form. Some publishers do not even have linked records for their entire archive.


\section{Case Study: Time Vault}

\section{Case Study: Vox cards? Quartz obsessions?}