\chapter{Conclusion}

\section{News apps}

% Also look at Skye Doherty's foray into design of hypernarratives in journalism

News archives form a symbiotic relationship with data-driven news apps, whether they exist for a specific event (think World Cup coverage or election results), or exist as standalone platforms (such as Homicide Watch, Syria Deeply, or Timelines). Such apps bypass the URL, but in the process they have their own challenges when saving and archiving. Scott Klein, luminary of news apps at the \emph{Times}, brings up Adrian Holovaty's ChicagoCrime.org, which Holovaty described as ``one of the original map mashups.''\autocite{holovaty_memory_2008} Launched in 2005, it is now defunct and unreachable in its original form; while the data survives, we have lost the presentation, and more importantly, the work, process, and context behind it.

There's no doubt that software constantly races against obsolescence, and some apps must be retired when their event passes or their function is done. But the lost process and context is lost knowledge. In March 2014, a group of NICAR conference attendees gathered at the Newseum to brainstorm the challenges and potentials of preserving news apps, suggesting more collaboration with libraries, museums, and cultural heritage institutions.\autocite{_opennews/hackdays/archive_????} Some such institutions are offering novel ways of preserving and maintaining digital work for the future. At the Cooper--Hewitt Museum, a team led by Seb Chan has been preserving the previously for-profit Planetary app as ``a living object.'' For the Cooper--Hewitt, preserving an app is more like running a zoo than a museum: ``open sourcing the code is akin to a panda breeding program.''\autocite{chan_planetary:_2013} They'll preserve the original, but also shepherd the open-source continuation of app development, thereby protecting its offspring and suggesting new applications for old frameworks. While the Cooper--Hewitt is currently guarding Silicon Valley technology, overlaps and partnerships between newspapers and cultural heritage institutions could lead to similar experiments.

\section{HTML5, multimedia, annotation}



% Here I will propose a solution; first by analyzing the data structure of some newsroom APIs, then suggesting a way to link them.

% Some tools that are cool: FOLD, Timelines, Evernote Context, Media in Context apps, Chorus, etc.


% Big theme: use a "push" rather than "pull" method to enliven the archive. The Times Report: you shouldn't have to do deep searches to find relevant stories, new or old. Younger audiences: "if something is important, it will find me" (Inno Report p. 39). Also: "we still ask too much of readers — they must navigate a website and apps that are modeled on our print structure." (p. 26)