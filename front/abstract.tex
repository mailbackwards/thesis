% $Log: abstract.tex,v $
% Revision 1.1  93/05/14  14:56:25  starflt
% Initial revision
% 
% Revision 1.1  90/05/04  10:41:01  lwvanels
% Initial revision
% 
%
%% The text of your abstract and nothing else (other than comments) goes here.
%% It will be single-spaced and the rest of the text that is supposed to go on
%% the abstract page will be generated by the abstractpage environment.  This
%% file should be \input (not \include 'd) from cover.tex.
As the pace of publishing and the volume of content rapidly increases on the web, citizen journalism has threatened the traditional, 20th-century role of journalism and institutional newsmaking. Journalists and publishers are beginning to adapt and evolve to fit into the new news landscape, but the role of legacy media institutions, and even newer, digital-native outlets, is still in flux and under debate. In this thesis I propose a framework for considering the news institution of the digital era as a \emph{linked archive}, equal parts news provider and information portal, one that places historical context on the same footing as new content, and emphasizes the journalist's role as explainer and verifier. Informed by a theoretical understanding of the web's structural affordances and limitations, and especially by the untapped networking power of the hyperlink, publishers can offer an archive-oriented model of sustainable and scalable journalism. Drawing from theories in library and computer science, an archivally focused journalistic model can save time for reporters and improve the research and reading process for journalists and audiences alike, treating news items as part of a conversation rather than a static box or endless feed, and putting the past in fuller and richer dialogue with the present.